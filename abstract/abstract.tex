%%%%%%%%%%%%%%%%%%%%%%%%%%%%%%%%%%%%%%%%%
% Journal Article
% LaTeX Template
% Version 2.1 (Jan 18, 2024)
%
% This template originates from:
% https://www.LaTeXTemplates.com
%
% Author:
% Vel (vel@latextemplates.com)
%
% License:
% CC BY-NC-SA 4.0 (https://creativecommons.org/licenses/by-nc-sa/4.0/)
%
% NOTE: The bibliography needs to be compiled using the biber engine.
%
%%%%%%%%%%%%%%%%%%%%%%%%%%%%%%%%%%%%%%%%%

%----------------------------------------------------------------------------------------
%	PACKAGES AND OTHER DOCUMENT CONFIGURATIONS
%----------------------------------------------------------------------------------------

\documentclass[
	a4paper, % Paper size, use either a4paper or letterpaper
	10pt, % Default font size, can also use 11pt or 12pt, although this is not recommended
	unnumberedsections, % Comment to enable section numbering
	twoside, % Two side traditional mode where headers and footers change between odd and even pages, comment this option to make them fixed
]{LTJournalArticle}

\addbibresource{bibliography.bib} % BibLaTeX bibliography file
\renewcommand*{\bibfont}{\footnotesize}

\runninghead{} % A shortened article title to appear in the running head, leave this command empty for no running head

\footertext{\textit{RISC-V Summit Europe, Munich, 24-28th June 2024}} % Text to appear in the footer, leave this command empty for no footer text

\setcounter{page}{1} % The page number of the first page, set this to a higher number if the article is to be part of an issue or larger work

%----------------------------------------------------------------------------------------
%	TITLE SECTION
%----------------------------------------------------------------------------------------

\title{Accelerating software development for emerging ISA extensions with cloud-based FPGAs: \\ RVV case study}
% Article title, use manual lines breaks (\\) to beautify the layout

% Authors are listed in a comma-separated list with superscript numbers indicating affiliations
% \thanks{} is used for any text that should be placed in a footnote on the first page, such as the corresponding author's email, journal acceptance dates, a copyright/license notice, keywords, etc
\author{%
	Marek Pikuła\textsuperscript{1}, Marek Szyprowski\textsuperscript{1}
}

% Affiliations are output in the \date{} command
\date{\footnotesize\textsuperscript{\textbf{1}}Samsung R\&D Institute Poland}

% Full-width abstract
\renewcommand{\maketitlehookd}{%
	\begin{abstract}
		\noindent The RISC-V Vector Extension (RVV) promises an enhanced performance
    and power efficiency across various complex computational tasks. However,
    the efficient utilization of RVV demands careful consideration of the
    optimization approach. This article examines strategies for accelerating
    this process. Key challenges include assessing performance differences among
    algorithmic approaches and overcoming initial hardware constraints. FireSim
    provides a comprehensive solution by offering advanced software and hardware
    simulation capabilities. Utilizing FireSim, we started the process of
    enhancing source code with RVV instructions (called vectorization) for the
    pixman project. Our experience outlines the efficacy of a cloud-based FPGA
    simulation in expediting software development for emerging ISA extensions.
    Overall, FireSim facilitates faster iteration cycles and informed design
    decisions, benefiting individual developers and fostering collaboration in
    remote teams.
	\end{abstract}
}

%----------------------------------------------------------------------------------------

\begin{document}

\maketitle % Output the title section

%----------------------------------------------------------------------------------------
%	ARTICLE CONTENTS
%----------------------------------------------------------------------------------------

\section{Introduction}

RISC-V Vector Extension (RVV) is making its way towards mainstream use. It opens
up exciting avenues for software vectorization on RISC-V, promising improved
performance and power efficiency across various computational tasks ranging from
image processing to AI workloads. However, realizing its full potential demands
careful evaluation and optimization of software implementations. This article
delves into the nuances of accelerating software development for new ISA
extensions, with a spotlight on the RVV extension.

\subsection{Approaches to Vectorization}

One of the key challenges in the software RVV vectorization is assessing the
performance difference between different algorithmic approaches. Since the
ratification in 2021, RVV support has gradually come to the upstream toolchains.
Although the development progress of auto-vectorizers in both GCC and LLVM is
promising, some use cases will always require a manual approach. Unfortunately,
not many reliable sources give a “golden standard” guide to follow. Two notable
exceptions are RVVRadar\autocite{KSG:2022} and RVV benchmark from Camel
Coder\autocite{CC:2024}, but both require physical hardware or manual
performance calculations.

\subsection{Initial Hardware Constraints}

At the outset of any project involving emerging ISA extensions like RVV, there
is often a lack of readily available hardware for testing and validation. While
tools like QEMU provide means to check the validity of the algorithms, they may
not accurately reflect performance characteristics. When comparing scalar and
RVV implementations of the same algorithm, the recently released QEMU~8.2
performs worse on the latter. This can lead to suboptimal design choices based
on incomplete or inaccurate performance assessments.

\subsection{Role of Simulation Tools}

Simulation tools play a crucial role in software development for emerging ISA
extensions. While the Spike ISA simulator offers robust tracing and debugging
functionality, it lacks in the area of microarchitectural considerations, which
are particularly important for such a large and complex ISA extension like RVV.

Software HDL simulation tools provide detailed profiling capabilities but suffer
from slow performance, especially for complex designs. Classical FPGA
prototyping offers a middle ground but comes with high initial costs and limited
tracing capabilities.

\section{FireSim Simulation}

To address these challenges, FireSim\autocite{ISCA-50} offers a comprehensive
solution. FireSim extends the ChipYard\autocite{chipyard} project with advanced
software and hardware simulation capabilities, making it an ideal platform for
developing and testing RISC-V core implementations, along with external or
internal specialized accelerators.

\subsection{Key Features}

FireSim provides several key features tailored to software development for
emerging ISA extensions:

\begin{itemize}
  \item Ready-to-use processor and accelerator cores with pre-built
        \emph{FireMarshal} Linux environment.
  \item \emph{TracerV}: %\autocite{FirePerf}
        In-hardware tracing with out-of-band Flame Graph visualization.
  \item Support for automatic ILA insertion, assertion synthesis, and
        out-of-band performance counters.
  \item Multicore and multi-SoC setups (so-called \emph{supernode} simulation)
        with on-chip network connected to the host for complex workload
        scenarios.
\end{itemize}

\subsection{Cloud Deployment}

One of the notable advantages of FireSim is its support for cloud deployment,
specifically on AWS EC2~F1 instances with Xilinx UltraScale+ VU9P FPGA cards.
This approach offers a low entry barrier on both technical and financial levels,
allowing teams to provision FPGA resources quickly and pay only for the compute
time used. It also significantly decreases the risk of the entire operation.
There is no need to buy expensive FPGA hardware or license Vivado, as everything
comes in a pay-by-hour manner, which is especially important for
software-focused teams.

This approach might also appeal to multidisciplinary, remote teams. For example,
the hardware team can develop an SoC on local hardware they already have on-site
(FireSim also supports local targets) and immediately pass the work to a remote
software team. Overall, this can dramatically increase the speed of evaluation
and integration and allow for more dynamic, end-to-end product development.

\section{RVV Case Study}

We utilized FireSim to develop and optimize the RVV port for the
\emph{pixman}\footnote{\url{https://gitlab.freedesktop.org/pixman/pixman}}
project. First, we evaluated readily available RVV-capable cores. We started
with the \emph{Tenstorrent Ocelot}\autocite{ocelot} project (a RISCV-BOOM core
with integrated RVV accelerator), which already had ChipYard integration.
Unfortunately, the implementation was not directly synthesizable in the FireSim
setting. The second choice was \emph{PULP Ara}\autocite{Ara2020}, a coprocessor
for the CORE-V CVA6 core. We smoothly adapted the existing ChipYard CVA6 wrapper
and successfully built and simulated the SoC on AWS node, running a provided
Fedora image by following the comprehensive and, importantly, complete
documentation of the aforementioned projects, even though we had no prior
experience with AWS tools.

\subsection{Development Process}

In the initial phase of the pixman vectorization, we focused on the correctness
of the approach and not on performance. For this we used QEMU environment,
because, even with its suboptimal RVV implementation, it greatly outperforms any
other emulation solution. Another important benefit is the possibility of
working on a local machine without the need to provision AWS resources.

Once we worked out the general solution, we started profiling the code in the
FireSim environment. This way, we could reliably compare between different
optimizations and with the base scalar implementation. To do this, we used
detailed TracerV reports along with performance counter measurements. We also
experimented with different microarchitectural configurations to see the
implications for overall performance, which allowed us to make informed
decisions regarding algorithmic approaches --  optimal in both low- and high-end
configurations.

\section{Conclusions}

Our experience with FireSim demonstrates the potential of the cloud-based FPGA
simulation for accelerating software development for emerging ISA extensions
like RVV. By providing access to cost-effective, scalable hardware resources and
comprehensive simulation capabilities, FireSim enables faster iteration cycles
and more informed design decisions. This approach not only benefits individual
developers but also facilitates collaboration in remote teams, bridging the gap
between hardware and software development efforts.

%----------------------------------------------------------------------------------------
%	 REFERENCES
%----------------------------------------------------------------------------------------

\printbibliography % Output the bibliography

%----------------------------------------------------------------------------------------

\end{document}
